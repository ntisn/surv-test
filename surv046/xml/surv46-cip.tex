\documentclass{surv-l}

\begin{document}

\begin{copyrightpage}
\begin{center}
\textbf{EDITORIAL COMMITTEE}
\end{center}

\noindent\qquad\qquad\qquad Georgia Benkart,
Chair\qquad\qquad\qquad Howard Masur

\noindent\qquad\qquad\qquad Robert
Greene\qquad\qquad\qquad\qquad\qquad Tudor Ratiu

Financial support from the following is gratefully acknowledged:
Mary Washington College and the Navy-ASEE (American Society of
Engineering Education) via the Marine Corps Systems Command
Amphibious Warfare Directorate under the Marine Corps Exploratory
Development Program (MQIA PE 62131M).

\begin{center}
1991 \emph{Mathematics Subject Classification}. Primary
20M20,20M18;

Secondary 20M05, 05C60, 20M30, 20B30.
\end{center}

\noindent \textsc{Abstract}. With over 60 figures, tables, and
diagrams, the text is both an intuitive introduction to and
rigorous study of finite symmetric inverse semigroups. It turns
out that these semigroups enjoy many of the classical features of
the finite symmetric groups. For example, cycle notation,
conjugacy, cummutativity, parity of permutations, alternating
subgroups, Klein 4-group, Ruffini's result\index{Ruffini's result}
on cyclic groups, Moore's presentations of the symmetric and
alternating groups, and the centralizer theory of symmetric groups
are extended to more general counterparts in the semigroups
studied here. We also classify normal subsemigroups, study
congruences, and illustrate and apply an Eilenberg-style wreath
product. The basic semigroup theory is further extended to partial
transformation semigroups, and the Reconstruction
Conjecture\index{Reconstruction Conjecture} of graph theory is
recast as a Rees ideal-extension~conjecture.\\

\textbf{Library of Congress Cataloging-in-Publication Data}

Lipscomb, Stephen, 1944--

\quad Symmetric inverse semigroups/Stephen Lipscomb.

\qquad p. cm. --- (Mathematical surveys and monographs, ISSN
0076-5376 ; v. 46)

\quad Includes bibliographical references and index.

\quad ISBN 0-8218-0627-0 (alk. paper)

\quad 1. Inverse semigroups. I. Title. II. Series: Mathematical
surveys and monographs; no. 46.

QA182.L56\quad 1996\hfill 96-26968

512$'$.2---dc20\hfill CIP\\

\textbf{Copying and reprinting.} Individual readers of this
publication, and nonprofit libraries acting for them, are
permitted to make fair use of the material, such as to copy a
chapter for use in teaching or research. Permission is granted to
quote brief passages from this publication in reviews, provided
the customary acknowledgment of the source is~given.

Republication, systematic copying, or multiple reproduction of any
material in this publication (including abstracts) is permitted
only under license from the American Mathematical Society.
Requests for such permission should be addressed to the Assistant
to the Publisher, American Mathematical Society, P. O. Box 6248,
Providence, Rhode Island 02940-6248. Requests can also be made by
e-mail to \texttt{reprint-permission@ams.org}.

\begin{center}
\textcopyright\ 1996 by the American Mathematical Society. All rights reserved.

The American Mathematical Society retains all rights except those granted to the United States Government.

Printed in the United States of America.

\circledinfty\ The paper used in this book is acid-free and falls within the guidelines established to ensure permanence and durability.

10\ 9\ 8\ 7\ 6\ 5\ 4\ 3\ 2\ 1\qquad 01\ 00\ 99\ 98\ 97\ 96
\end{center}
\end{copyrightpage}
\end{document}
